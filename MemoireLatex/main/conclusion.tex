\part*{Conclusion}
\addcontentsline{toc}{part}{Conclusion}
\markboth{Conclusion}{Conclusion} 

Nos deux stages nous ont permis d'approfondir les enjeux de l'édition numérique de correspondance par deux approches différentes, sur des corpus du XIX\up{e} siècle. Cette double expérience nous a permis d'avoir une vue plus large et de découvrir différents projets auxquels nous avons pris plaisir à participer.

Dans une première partie, nous nous sommes penchée sur les contextes des projets, les buts qu'ils se proposent, et la matière première si l'on peut dire, sur laquelle ils se basent pour atteindre les objectifs fixés. Ces éléments sont essentiels pour avoir une meilleure vue d'ensemble des projets et cerner ensuite quel rôle sera le nôtre, comment nous nous inscrivons dans ces projets, et quels moyens prendre pour y contribuer et les faire avancer .
Nous avons donc analysé le contexte universitaire et intellectuel des projets : notre but est de faire avancer la recherche historique et culturelle en général. Cependant, ces deux projets, bien que travaillant tous deux à la diffusion de correspondances du XIX\up{e} siècle, diffèrent dans leur but. Certes, ils ont les points commun du genre épistolaire et du cadre dans lequel les lettres ont été produites, le XIX\up{e} siècle, mais le CRHXIX \oe uvre à une édition plus classique, centrée sur une personne autour de laquelle se greffe un réseau de correspondants, le but restant la recherche historique, alors que pour le Labex OBVIL, ELICOM est un moyen de pousser les possibilités du numérique en terme de fouilles de données, d'enrichissement et de visualisation, ainsi que la recherche en humanités numériques, avec un séminaire de recherche consacré au projet et aux avancées de l'édition numérique de correspondance.
Bien sûr, certains outils restent les mêmes, mais l'optique est différente.
Cette différence s'accentue lorsque l'on considère les sources qui servent de point de départ aux deux projets. Pour le CRHXIX, nous avons affaire à des manuscrits originaux, qui ont été numérisés ou devront l'être.
Certes, nous n'avons pas accès directement à la source mais à son fac-similé. Nous sommes néanmoins assez proches de l'original.
Par ailleurs, nous ne nous appuyons aucunement sur des éditions précédentes, en ce sens-là, notre travail d'édition est conséquent puisque nous ouvrons la voie. Pour ELICOM, au contraire, nous travaillons sur des sources indirectes. Nous n'avons pas accès aux manuscrits ni à des fac-similés mais à de précédentes éditions imprimées. Nous prenons de la distance avec elles dans les notes et certains choix, mais elles restent tout de même notre point de départ. Notre rapport à la source est donc différent, dans l'un et l'autre projet. Or, la source a son rôle dans le choix des moyens car on les adapte à elle. Ce sont ces considérations qui nous ont occupée pendant la première partie. 

Après avoir exposé nos projets et leurs différentes facettes, nous avons souhaité prendre un peu de hauteur par rapport à l'édition numérique de correspondance aujourd'hui, et faire un état de l'art : où en sommes-nous, quels sont les moyens mis à notre disposition ? 
Toutes ces réflexions vont influencer notre manière d'agir et d'aborder nos projets. Nous avons donc établi un petit bilan scientifique sur les réflexions de la communauté scientifique autour de l'édition numérique de correspondance, et les divers déploiements d'outils pour favoriser son essor. Nous avons abordé les problématiques propres à l'édition numérique de correspondance, sachant que l'aspect numérique est essentiel. 
Nos considérations sur l'édition numérique de correspondance se sont conclues par la pratique : nous avons donc envisagé notre futur site pour le CRHXIX, concluant ainsi notre deuxième partie.

Après la théorie, nous sommes donc passée à la pratique, tout d'abord pour ce qui est de l'acquisition des données, ce qui a fait l'objet de notre troisième partie. Or, ici, nous avons constaté que pour nos deux projets, la nouvelle technologie de l'apprentissage machine s'est avérée être au c\oe ur de l'acquisition des données, tant pour ELICOM via l'OCR de Gallica, que pour notre projet d'édition numérique de la correspondance de Frédéric Le Play, via l'HTR de Transkribus. Nous avons particulièrement développé ce point de Transkribus, outil de transcription collaborative fort utile. Néanmoins, des interrogations subsistent quant à la rentabilité du modèle, au traitement et au mode d'importation des données. Vaudrait-il mieux passer par XSLT ? Devrait-on penser davantage au \emph{tags} dans Transkribus et les personnaliser ? Ce sont autant de questions qui se posent encore. Quant à l'OCR de Gallica, nous avons pu voir qu'il conditionnait beaucoup le pré-traitement et traitement des données sur lequel nous nous sommes plus attardée en quatrième partie.

Après avoir considéré l'acquisition des données, nous nous sommes penchée sur leur traitement. Dans cette partie, nous avons vu combien le langage XML est utilisé dans les éditions numériques de correspondance. Celui-ci qui a, en soi, de multiples possibilités, est restreint en l'occurrence à nos besoins d'édition de correspondance. Nous avons donc expliqué nos choix de balises, leur documentation via l'ODD, ainsi que les technologies utilisées pour pousser au maximum les possibilités d'XML, à savoir XSLT - qui reste encore un moyen à utiliser éventuellement - et Python. Puis nous avons exposé les difficultés inhérentes aux projets numériques, les défis à relever, et les moyens que nous avons employés dans ce but, pour que les projets soient menés à terme. Nous n'avons été qu'un maillon de la chaîne, mais il ne doit pas manquer, au risque de briser cette chaîne.

Le contexte de télétravail a rendu l'expérience de ces stages particulièrement inédite. Nous avons dû relever le défi de travailler seule et de n'être reliée aux différentes équipes que par le net. Nous avons en quelque sorte travaillé avec des équipes qui nous ont paru un peu virtuelles, quoique toujours là pour répondre à nos questions.

Malgré tout, le bilan reste très positif. Nous avons pu mettre en pratique les connaissances reçues à l'ENC, et nous familiariser toujours plus aux diverses technologies. Nous avons particulièrement apprécié le fait de voir concrètement, dans des projets bien réels, les avantages et les possibilités du numérique. 

Pour ce qui est de notre stage au Labex OBVIL, au total, 388 lettres ont été extraites, dont 264 fichiers ont été corrigés durant ces 20 jours de stage. Nous avons donc eu la satisfaction d'avoir pu participer au projet ELICOM.
Dans l’ensemble, ce stage nous a bien aidée à consolider nos connaissances en XML et nous a indirectement aidée pour le projet Le Play, grâce à leur point commun d'édition de corpus épistolaires du XIX\up{e} siècle. Le plus grand défi de ce stage s'est situé dans la rédaction du code Python que nous avons appris à mieux maîtriser, même si de grands progrès restent à faire.

Quant à notre stage au CRHXIX, il a été extrêmement enrichissant car nous avons été chargée de penser la partie numérique du projet. Durant tout le confinement, nous avons été la seule personne de l'équipe, à quelques exceptions près, à pouvoir nous consacrer totalement à ce projet.
Sans les remarques du chef de projet et de membres de l'équipe, et sans l'aide des professeurs de l'École, particulièrement celle de notre tuteur, ainsi que le soutien de notre tuteur d'OBVIL, nous n'aurions pas pu aller aussi loin dans le projet. Durant une petite trentaine de jours, nous avons dû faire nôtre les enjeux du projet, prendre en main Transkribus qui nous était presque totalement étranger, entraîner un modèle avec plus de 20 000 mots, penser à l'exportation en XML-TEI, se faire une idée du futur site et en fonction, penser l'encodage en TEI. Cela a donc été un stage particulièrement riche par sa diversité. Il n'a pas été exempt de limites : nous n'avons pu réaliser de véritable cahier des charges ni de récits utilisateurs dignes de ce nom. Néanmoins, il était difficile de tout mener de front en si peu de temps.

Ces différentes approches de l'édition numérique de correspondance ont donc été très enrichissantes.
Nous avons pu constater que l’épistolaire à l’ère du numérique se situe réellement entre des perspectives historiques, ou plus largement des perspectives de sciences humaines, d'aide à la recherche et autres, et des innovations technologiques. Nous sommes entre les sciences humaines et le numérique, nous sommes tout simplement dans les humanités numériques, avec peut-être un aspect plus \inquote{humanités} pour le projet Le Play, et plus \inquote{numérique} pour ELICOM. 

Ainsi, ces projets sont l'illustration que les humanités numériques, comme le souligne le \emph{Manifeste des digital humanities}\footnote{\emph{Manifeste des digital humanities}, THATCamp, Paris, 2010 URL : \url{https://tcp.hypotheses.org/318} (visité le 29/09/2020)}, \inquote{ne font pas table rase du passé, mais s'appuient, au contraire, sur l'ensemble des paradigmes, savoir-faire et connaissances propres à ces disciplines, tout en mobilisant les outils et les perspectives singulières du champ du numérique}.