\part*{Introduction}
\addcontentsline{toc}{part}{Introduction}
\markboth{Introduction}{Introduction}
\hugeskip

\vspace*{\fill} 
\epigraph{\itshape \inquote{Pratiquer l’édition numérique signifie prendre en compte ce lien étroit entre la technique et la culture.}}{Michaël E. Sinatra \\ \emph{Pratiques de l'édition numérique}}
\vfill\clearpage
\cleardoublepage

A la fin du XX\up{e} siècle apparaît le terme de \inquote{révolution numérique}, désignant les mutations profondes des sociétés dues à l'essor des nouvelles technologies numériques.
Cette révolution atteint tous les niveaux de la société \footnote{Nous ne traitons pas ici de la fracture numérique qui pose des limites à cette affirmation. La révolution numérique n'en est pas moins une réalité.} et le monde universitaire et culturel connaît également des changements. Ainsi constate-t-on que depuis plusieurs années,

\begin{quotation}
\inquote {les chercheurs en humanités et sciences sociales vivent une transformation radicale de leur travail. Ces érudits que l’on imagine volontiers enfouis sous des piles de livres, fouillant des masses de vieux papiers dans les archives, [...] passent aujourd’hui le plus clair de leur temps sur leur… ordinateur. Comme le médecin, l’avocat ou le journaliste, le chercheur contemporain a vécu en moins de vingt ans, c’est-à-dire même pas l’espace d’une génération, une dématérialisation pratiquement complète des conditions d’exercice de son métier\footnote{Pierre Mounier, \inquote{Les Humanités numériques, gadget ou progrès ? Enquête sur une guerre souterraine au sein de la recherche}, \emph{Revue du Crieur}, 2017/2, p. 144-159, URL : \url{https://www.cairn-int.info/revue-du-crieur-2017-2-page-144.htm} (visité le 02/09/2020)}.}
\end{quotation}

L'on assiste à la naissance des humanités numériques ou \emph{digital humanities}. En effet, comme le souligne le \inquote{Manifeste des \emph{digital humanities\footnote{Manifeste rédigé par les participants du THATCamp Paris de mai 2010. Voir Michaël E. Sinatra et Marcello Vitali-Rosati, \emph{Pratiques de l'édition numérique}, Montréal, Les Presses de l'Université de Montréal, 2014}}}, on remarque que \inquote{le tournant numérique pris par la société modifie et interroge les conditions de production et de diffusion des savoirs.} 

Cette dématérialisation ne peut se faire sans l'aide de personnes qui améliorent sans cesse les outils de recherche et l'offre proposée en ligne, sur les bibliothèques numériques ou les sites abritant des savoirs, permettant ainsi leur diffusion, leur partage et leur valorisation. Une de ces applications significatives est l'édition numérique de correspondance. Notre sujet se trouve donc au c\oe ur des humanités numériques.

Avant tout, il faut savoir que l'éditeur d'une édition papier est chargé du choix des contenus, de leur légitimation et de leur diffusion. 
L'édition numérique garde ces caractéristique et se double d'un
\begin{quotation}
\inquote{ensemble complexe de pratiques qui vont bien au-delà du rôle que l’éditeur a eu dans le modèle de l’édition imprimée à partir du XVIII\up{e} siècle. [Elle] regroupe toutes les actions destinées à structurer, rendre accessible et visible un contenu sur le web\footnote{Michaël E. Sinatra et Marcello Vitali-Rosati, \emph{Pratiques de l’édition numérique}, Presses de l’Université de Montréal, 2014, URL : \url{https://books.openedition.org/pum/308} (visité le 03/09/2020)}.}
\end{quotation}

Elle a donc recours à nombre de pratiques, de traitements et d'outils technologiques avancés pour assurer à la fois une édition de qualité quant au fond et à la forme, et la pérennité des données.

Au sein de l'édition numérique en général se trouve une édition plus spécifique qui représente à elle seule un monde à part : c'est l'édition numérique de correspondance. 

La correspondance a elle-même ses caractéristiques : c'est un genre à la fois protéiforme - c'est à dire qu'elle est adressée à un public restreint ou étendu - réticulaire - elle se trouve dans un réseau de lettres et de correspondants - et elliptique - d'où l'importance d'une documentation conséquente\footnote{ Richard Walter (dir.), \emph{L’édition numérique de correspondances – guide méthodologique}, 2018, p.4 URL :  \url{https://cahier.hypotheses.org/guide-correspondance} (visité le 17/06/2020)}.
L'édition numérique de correspondance doit donc prendre en compte toutes ces caractéristiques et tirer profit de l'aspect numérique pour enrichir les possibilités d'une édition classique.

Lors de notre stage, nous avons donc été confronté à toutes ces problématiques autour de l'édition numérique de correspondance.
Notre réflexion a eu pour cadre deux institutions dépendant de la Sorbonne (Sorbonne Université et Paris I), à savoir le Labex OBVIL (Observatoire de la vie littéraire) et le Centre de Recherche d'Histoire du XIX\up{e} siècle (CRHXIX). Chacune des institutions menant son propre projet, nous avons donc travaillé sur deux projets distincts. 

Au Labex OBVIL, nous avons pris part au projet de plateforme multifonctionnelle ELICOM, le nom ELICOM étant l'abréviation de Éditer, Lire des Correspondances Multidisciplinaires : l'appellation du projet fait entrevoir à lui seul ses buts et son ampleur. Il répond à trois objectifs : la collecte et l’enrichissement de correspondances épistolaires de différentes disciplines, la fouille transversale des données et métadonnées, et l’exploitation visuelle et statistique des résultats de cette fouille. 
Par ailleurs, les corpus de correspondances épistolaires sont multidisciplinaires, c'est à dire à la fois littéraires, philosophiques et scientifiques, et tous appartiennent au XIX\up{e} siècle.
C'est donc un projet innovant et au carrefour de nombre de standards, technologies et outils.

Le CRHXIX quant à lui porte un projet d'édition numérique à caractère plus classique, autour de la correspondance du sociologue Frédéric Le Play (1806-1882). Nous sommes encore sur un corpus du XIX\up{e} siècle, mais qui nécessite d'être traité différemment. Au Labex OBVIL, nous partons de différentes éditions papier pour aller vers une édition numérique de grande ampleur. Au CRHXIX au contraire, nous partons des sources manuscrites pour réaliser une édition numérique faite de documents inédits. Cette différence conséquente entraîne un traitement autre et l'utilisation d'outils spécifiques pour les transcriptions.

Certes, à des besoins différents répondent l'emploi d'outils distincts et d'un traitement approprié à chacune des éditions.
Néanmoins, nombre de problématiques sont communes à ces éditions numériques de correspondance, car nous restons dans un même sujet. Deux grands axes rejaillissent à travers l'acquisition et le traitement des données.

Tout d'abord, nous sommes arrivés aujourd'hui à l'ère de l'intelligence artificielle (IA)\footnote{L’IA est un ensemble de techniques permettant à des machines d’accomplir des tâches et de résoudre des problèmes normalement réservés aux êtres humains.}, et autour de cette réalité gravitent nombre de technologies. Dans l'édition numérique de correspondance, l'on a ainsi recours à l'apprentissage machine ou \emph{machine learning} dans la phase d'acquisition des données, que ce soit via la Reconnaissance optique de caractères dite OCR (\emph{Optical Character Recognition}), ou même via la technologie plus avancée qu'est la Reconnaissance de l'écriture manuscrite dite HTR (\emph{Handwritten Text Recognition}).

D'autre part, quant au traitement des données, l'édition numérique de correspondance utilise ultimement le standard XML-TEI, comme l'a souligné assez récemment l'étude du Consortium CAHIER dans son \emph{Guide méthodologique} consacré à ce sujet\footnote{ Richard Walter (dir.), \emph{L’édition numérique de correspondances – guide méthodologique}, 2018 URL :  \url{https://cahier.hypotheses.org/guide-correspondance} (visité le 17/06/2020)}.

L'objectif de ce mémoire sera donc de voir, à travers ces deux projets qui portent sur des corpus d'un même siècle mais diffèrent parfois dans leur méthodologie et outils, les grandes tangentes et les possibilités qui caractérisent l'édition numérique de correspondance aujourd'hui.

Pour cela, il nous faudra tout d'abord nous pencher, dans une première partie, sur les projets en eux-mêmes, pour mieux comprendre leurs enjeux, leurs besoins, leurs objectifs, afin de pouvoir y répondre.

Or, l'on ne pourra y répondre sans avoir, au préalable, mené une réflexion conséquente sur ce qu'implique l'édition numérique de correspondance aujourd'hui. Il faudra s'interroger sur ses moyens et ses outils et prendre du recul face à ces problématiques pour mieux les appliquer à nos projets. Penser l'édition numérique de correspondance sera donc l'objet de notre deuxième partie.

 Ces réflexions nous amèneront à leur mise en pratique qui se fera en deux temps. Tout d'abord, notre troisième partie sera consacrée aux moyens employés pour l'acquisition des données, au c\oe ur de laquelle se trouve l'apprentissage machine, sans oublier d'autres technologies que nous développerons à cette occasion.
 
 Une fois les données acquises, il s'agira d'assurer leur traitement, ce qui fera l'objet de notre quatrième partie, qui abordera la question des standards et formats employés pour une meilleure pérennité des données, notamment XML-TEI.

