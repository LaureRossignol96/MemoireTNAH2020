\chapter*{Résumé}
\addcontentsline{toc}{chapter}{Résumé}
\markboth{Résumé}{}

Ce mémoire a été réalisé en vue de l'obtention du diplôme de Master 2 \inquote{Technologies numériques appliquées à l'histoire} de l'École nationale des chartes. Il a été rédigé suite à la réalisation d'un stage d'environ trois mois au sein de deux institutions dépendant de la Sorbonne, le Centre de Recherche d'Histoire du XIX\up{e} siècle, sur le projet d'édition numérique de la correspondance du sociologue Frédéric Le Play (1806-1882) d'une part, et le Labex OBVIL sur le projet ELICOM d'autre part. Ce travail n'est ni un mémoire de recherche, ni un rapport de stage. Il vise à apporter une analyse critique des enjeux, stratégies et résultats de ces projets qui ont en commun le siècle sur lequel ils se penchent, le type de sources, la volonté de valorisation et d'accessibilité, l'utilisation de certains outils numériques, mais qui diffèrent aussi sous certains rapports, et qui s'inscrivent dans le cadre des humanités numériques.\\

\medskip

%informations à compléter
\textbf{Mots-clefs:} apprentissage machine ; architecture et arborescence de site ; Centre de Recherche d'Histoire du XIX\up{e} siècle ; édition numérique de correspondance ; ELICOM ; Frédéric Le Play ; Gallica ; HTML ; HTR ; index ; Labex Obvil ; numérisation ; OCR ; ODD ; Python ; Relax NG ; SEO ; sociologie ; XML-TEI ; XSLT\\

% informations bibliographiques
\textbf{Informations bibliographiques:} Lucie Slavik, \textit{L’édition numérique de correspondance, à travers deux applications sur des corpus du XIX\up{e} siècle : la correspondance de Frédéric Le Play (CRHXIX) et ELICOM (Labex OBVIL)},  mémoire de master \og Technologies numériques appliquées à l'histoire \fg{}, dir. Thibault Clérice et Arthur Provenier, École nationale des chartes, 2020.

\clearpage
\thispagestyle{empty}
\cleardoublepage
