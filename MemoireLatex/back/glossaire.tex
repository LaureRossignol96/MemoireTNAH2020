\chapter*{Glossaire}
\addcontentsline{toc}{chapter}{Glossaire}
\markboth{Glossaire}{Glossaire}
\pagestyle{myheadings}

\begin{itemize}
    \item \textbf{Bibliothèque numérique} : Ensemble organisé de documents nativement numériques ou numérisés accessibles à distance par internet.
    \item \textbf{Format} : Manière normalisée de représenter des données ou des fichiers sous la forme d'informations binaires.
    \item \textbf{ID} : \emph{IDentifier}, abréviation anglaise de identification ou identifiant, il sert à identifier un objet précis dans un ensemble d'objets.
    \item \textbf{Interface graphique} : Par opposition à l'interface en ligne de commande, l'interface graphique désigne la manière dont est présenté un logiciel à l'écran pour l'utilisateur, permettant l'interaction entre l'humain et la machine. Une interface graphique bien conçue est ergonomique et intuitive afin que l'utilisateur la comprenne tout de suite.
    \item \textbf{Module} : En Python, un fichier pouvant contenir des fonctions, des classes et des données, et pouvant être importé dans un script.
    \item{\textbf{ODD}\textit{(One Document Does it all)}}:  Langage de définition et de maintenance dusystème TEI. Il permet la maintenance du code et de sa documentationd’une manière intégrée, à partir d’une seule source XML. Il se compose d'un schéma formel (utilisant un langage informatique tel que DTD, RELAX NG, W3C Schema, Schematron) pour contrôler l’édition et d'une documentation explicant aux utilisateurs ou développeurs les principes éditoriaux et choix de balises.
    \item \textbf{Python} : Langage de programmation informatique à usage général, multi-plateforme et \emph{open-source}.
    \item \textbf{\emph{Package}, \emph{library}} : Un ensemble de modules contenant des outils tels que des fonctions. Pour être utilisé, il doit être importé entièrement ou partiellement, par module.
    \item \textbf{Parser} : Processus d'analyse d'un élément textuel le rendant intelligible par la machine, sous la forme d'un encodage numérique.
    \item \textbf{\emph{Script}} : Un script désigne un programme, entier ou extrait, chargé d'exécuter une action prédéfinie quand un utilisateur réalise une action ou qu'une page web est en cours d'affichage sur un écran. Il s'agit d'une suite de commandes simples et souvent peu structurées qui permettent l'automatisation de certaines tâches successives dans un ordre donné. 
    \item \textbf{Standard} : Texte de référence reconnu, documenté et élaboré par un groupe de travail spécialisé, visant à harmoniser l'activité d'un secteur donné. Pour XML, les standards prennent la forme de schémas et de règles de balisage permettant de créer des documents de structures comparables au sein d'un même standard.
    \item \textbf{TEI} : \emph{Text Encoding Initiative} - Standard de description de documents textuels pour XML. Développé par le TEI Consortium.
    \item{\textbf{Transkribus}} : Plateforme de transcription automatique de textes manuscrits. Fondé sur l’intelligence artificielle, le moteur de reconnaissance de texte manuscrit (Handwritten Text Recognition ou HTR) doit être préparé avec des donnéesd’apprentissage, obtenues par la transcription d’une centaine de pages minimum, en établissant la correspondance ligne à ligne entre l’image du texte numérisé et sa transcription.
    \item{\emph{\textit{\textbf{User Story}}}} : Un récit utilisateur, ou «\textit{ user story} »  en anglais, est une description simple d’un besoin ou d’une attente exprimée par un utilisateur et utilisée dans le domaine du développement de logiciels et de la conception de nouveaux produits pour déterminer les fonctionnalités à développer.
    \item \textbf{VIAF} : \emph{Virtual International Authority File} ou Fichier d'autorité international virtuel, est un fichier d'autorité international servant à identifier les personnes ou les collectivités contenues dans d'autres fichiers d'autorité
    \item \textbf{Wiki} : Application web dont le contenu peut être édité par les visiteurs, ce qui permet la création et la modification des pages de manière collaborative. Il est généralement dédié à un projet ou à une thématique précise.
    \item \textbf{XML} : \emph{eXtensible Markup Language} -  Un langage de balisage générique permettant de décrire des informations de manière organisée et standardisée. Le XML est une recommandation du W3C [\url{https://www.w3.org/XML/}].
    \item \textbf{XSLT} : \emph{Extensible Stylesheet Language Transformations} - Un langage basé sur XML permettant de styliser ou transformer des fichiers XML ou HTML.
\end{itemize}